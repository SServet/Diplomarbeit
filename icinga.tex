\chapter{Servicedefinierung}
\label{chap: Servicedefinierung}

\section{Möglichkeiten}
\label{chap: Möglichkeiten zur Servicedefinierung}

Alle Unternehmen stellen Dienstleistungen zu Verfügung. Doch damit dies auch in angemessener Qualität zustande kommt, 
müssen Unternehmen permanent agil und anpassungsfähig sein. Durch die Definierung der IT-Services, mittels einer Ansetzung an Best Practices, 
ist die Erhöhung der Erfolgschancen deutlich höher.
\

Dabei bieten diese Quellen Best Practices:

\begin{quote}
    \begin{itemize}
        \item   \emph{Öffentliche Frameworks und Standards
        \item   Proprietäres Wissen von Organisation und einzelnen Personen}
    \end{itemize}
\end{quote}

so laut \citetitle[]{ITIL}\footcite[][Kap.\ 1.1]{ITIL}

\noindent
\newline
In diesem Fall wird für die erste Möglichkeit sich entschieden,
weil innerhalb des Unternehmens SSIT Solutions KG kein qualifiziertes
 Wissen bezüglich des IT Service Managements vorhanden ist.
 Dabei stellt sich nur noch in Frage, welches der Frameworks der momentanen Situation
 sich eignet.

\subsection{Vergleich zwischen diversen Frameworks}
\label{chap: Vergleich Frameworks}

Es gibt eine Menge an öffentlichen Frameworks, die von verschiedenen Publikatoren
veröffentlicht wurden. Jedoch drei von ihnen sind in diesem Fall interessant 
(absteigend sortiert nach Häufigkeit der Verwendung):
\newline
\begin{itemize}
    \item \textbf{ITIL}:  \textbf{I}nformation \textbf{T}echnology \textbf{I}nfrastructure
                          \textbf{L}ibrary
    \item \textbf{COBIT}: \textbf{C}ontrol \textbf{Ob}jectives for \textbf{I}nformation 
                          and Related \textbf{T}echnologies
    \item \textbf{MOF}:   \textbf{M}icrosoft \textbf{O}perations \textbf{F}ramework
\end{itemize}
\noindent
\newline
Zwar sind alle Frameworks nutzvoll, doch das sind sie erst, wenn sie in der richtigen
Ausgangslage verwendet werden. 
\newline
Hier eine kurze Darstellung der Kernprozesse der erwähnten Frameworks:

\begin{itemize}
    \item ITIL
    \begin{itemize}
        \item Service Strategy
        \item Service Design
        \item Service Transition
        \item Service Operation
        \item Continual Service Improvement
    \end{itemize}
    \item COBIT
    \begin{itemize}
        \item Plan and Organise
        \item Acquire and Implement
        \item Deliver and Support
        \item Monitor and Evaluate
    \end{itemize}
    \item MOF
    \begin{itemize}
        \item Plan
        \item Deliver
        \item Operate
        \item Manage
    \end{itemize}
\end{itemize}

\noindent
\newline
COBIT wird dabei eher für Unternehmen verwendet, die hauptsächlich
 eine Prüfung der IT-Prozesse durchführen. Dabei sind COBIT und ITIL zwar ziemlich
 ähnlich, aber der Hauptunterschied liegt dabei, dass COBIT prozessbasiert ist.
 ITIL hingegen ist dienstleistungsbasiert.{\footnote{\url{https://de.wikipedia.org/wiki/COBIT}}
\newline
MOF ist verglichen mit ITIL allerdings basierend auf Service Management Funktionen
und geeignet für alltägliche IT Praktiken, wobei beide aber einem Lebenszyklus 
der Dienstleistungen folgen.\footnote{\url{https://technet.microsoft.com/en-us/library/dd320379.aspx}}

