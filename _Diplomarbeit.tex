\documentclass[german,oneside,color]{htldipl}
% Zulässige Class Options: 
%   Hauptsprache: german (default), english
%   Doppelseitig: oneside (default), twoside
%   Syntax-Highlighting: color (default), black

% die folgende Zeile einkommentieren für Arial-Ähnliche Schriftart
%\renewcommand{\familydefault}{\sfdefault}

\graphicspath{{images/}}    % wo liegen die Bilder?
\usepackage[paper=a4paper,margin=3cm]{geometry}

\makeglossaries
%\include{glossary}
\loadglsentries{glossary}


\addbibresource{literatur.bib}     %BibTeX-Datei literatur.bib

%%%----------------------------------------------------------
\begin{document}
%%%----------------------------------------------------------

\title{Automatisierte Datensammlung in mehreren Netzwerken}
\abteilung{Informatik}
%\schwerpunkt{} wenn kein Ausbildungsschwerpunkt vorhanden ist z.B. Informatik
\schwerpunkt{}
\studienort{Wiener Neustadt}
\schule{HTBLuVA Wiener Neustadt}
\schullogo{htl.jpeg}
\abgabejahr{2017/18}
\betreuerA{Ing. -Mag. -Dr. \ Werner Birkelbach-Baumgartner}
\betreuerB{}
\betreuerC{}
\betreuerD{}
%\betreuerD{} leer lassen wenn nicht vorhanden
\schuelerA{Maximilian MAIER}
\evidenzA{5AHMIA-17}
\subthemaA{Konstruktion des Versuchstandes}
\schuelerB{Caner Demirbag}
\evidenzB{5AHMIA-19}
\subthemaB{Erstellen der Pumpenkennlinien}
\schuelerC{Simsek Servet-can}
\evidenzC{5AHMIA-24}
\subthemaC{Integration des Versuchstandes in die bestehende Softwarelösung für die Kennlinien}
\schuelerD{}
\evidenzD{}
\subthemaD{}
\schuelerE{}
\evidenzE{}
\subthemaE{}
%\schuelerE{} leer lassen wenn nicht vorhanden
%\evidenzE{}
%\subthemaE{}

%%%----------------------------------------------------------
\frontmatter
\maketitle
\tableofcontents
%%%----------------------------------------------------------

\chapter{Vorwort}

An dieser Stelle bedanken wir uns bei unserem begleitenden Professor Ing. Mag. Dr. Werner Birkelbach-Baumgartner,
für seine hilfreiche Anleitung und seine Unterstützung während dieses Prozesses.                  

\SuperPar Im Weiteren bedanken wir uns an unserem Abteilungsvorstand Dipl.-Ing. Felix Schwab und dem Netzwerkadministrator unserer Schule,
die uns die Technologien Icinga und Nagios nahe gebracht haben.                 

\SuperPar Außerdem bedanken wir uns bei unserem Auftraggeber, SSIT-Solutions KG die für uns die notwendigen Ressourcen zur Verfügung gestellt haben,
die für die Realisierung dieses Projektes notwendig waren.                     

\SuperPar Zum Abschluss bedanken wir uns, an unseren Familien die uns über das ganze Schuljahr motiviert haben.






				%ggfs. weglassen
\include{dokumentation}
\chapter{Kurzfassung}

Im Rahmen der Diplomarbeit wird für das Unternehmen 
SSIT Solutions KG ein Monitoringsystem ausgesucht.
Es wurden zwei Möglichkeiten zur Überwachung der Netzwerke 
verglichen und Icinga wurde als Monitoring-Tool ausgewählt.
MIt der Hilfe von Icinga und einer MSSQL-Datenbank werden die Kundennetzwerke 
und deren Geräte verwaltet.

Mittels ITIL wird eine Messung und Überwachung
von IT-Services und Service Providern erzielt, 
sowie das Management von Risiken, Fähtigkeiten und Ressourcen zur 
effektiven und effizienten Serviceerbringung.		
\chapter{Abstract}

\begin{english} 
    For the company SSIT Solutions KG we are 
    choosing a monitoringsystem in the context of our diploma thesis.
    We compared two tools for monitoring the network and decided to choose Icinga.
    With the help of Icinga and a MSSQL Database the client 
    networks and their devices will be administrated.

    Through ITIL is a measurement and a monitoring of IT services and 
    service providers achieved. As like as the management of risks, abilities 
    and resources for a effective and efficient service delivery.
\end{english}			

%%%----------------------------------------------------------
\mainmatter           %Hauptteil (ab hier arab. Seitenzahlen)
%%%----------------------------------------------------------

\include{einleitung}
\include{diplomschrift}
\include{latex}
\include{faq}
\include{abbildungen}
\include{mathematik}
\include{literatur}
\include{drucken}
\include{word}
\include{schluss}

%%%----------------------------------------------------------
%%%Anhang
\appendix
\printglossaries
\include{anhang_a}	% Technische Ergänzungen
\include{anhang_b}	% Inhalt der CD-ROM/DVD
\include{anhang_c}	% Chronologische Liste der Änderungen
\include{anhang_d}	% Quelltext dieses Dokuments

%%%----------------------------------------------------------

%Literaturverzeichnis
\clearpage
\addcontentsline{toc}{chapter}{\bibname}

\printbibliography










%%%----------------------------------------------------------

%%%Messbox zur Druckkontrolle
\include{messbox}

\end{document}
