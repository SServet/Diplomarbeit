\chapter{Aufbau der Testumgebung}
\label{chap: Testumgebung}

\section{Virtuell oder Pysisch?}
\label{sec: VirtuellReal}

Am Anfang wurde diskutiert ob die Testumgebung (in diesem Fall ist die Testumgebung das Netzwerk wo wir das Monitoring durchführen werden) 
virtuell oder pysisch sein soll. Nach einem kurzen Meinungsaustausch aller Beteiligten wurde es entschieden dass beide Varianten verwendet werden.
 Der Grund dafür war dass wir nicht zu jedem Zeitpunkt Zugang zu der pysischenn Testumgebung hätten, aber die virtuelle Umgebung alleine nicht ausreichen 
 würde da man zum Beispiel keinen Netzwerkdrucker direkt in ein virtuelles Netzwerk integrieren kann.
 Es wurde mit dem Auftraggeber ausgemacht dass wir uns selber um das virtuelle Netzwerk kümmern und dass wir die pysische Umgebung von 
 ihm zur Verfügung gestellt bekommen.


 \section{Möglichkeiten zur Virtualisierung}
\label{sec: Verglichene Virtualiesierungsplattformen}

 Gleich am Anfang der Arbeit wurden die vier bekanntesten Virtualiesierungsplattformen vergliechen. 
 Die Plattformen die vergliechen wurden sind:

\begin{itemize}
         \item Virtualbox
         \item VMware Workstation Player
         \item VMware Workstation Pro
         \item Hyper-V
\end{itemize}

 \section{Gegenüberstellung der Virtualisierungsmöglichkeiten}
\label{sec: Gegenüberstellung der Virtualisierungsmöglichkeiten}

\textbf{Virtualbox:} Die Open-Source-Software ist durch ihre umfangreiche Ausstattung sowie ihrer 
guten Bedienerführung eines der meisten benutzten Tools zur Virtualisierung. Virtualbox unterstützt 32- 
und 64-Bit-Maschinen, die zwei gängingsten USB Standards 2.0 und 3.0, sowie USB-Laufwerke die man als 
virtuellen Datenspeicher einbinden kann. Mit dem letzten Update kam auch ein bilddirektionales Drag\&Drop 
für die Gastsysteme. Unter Virtualbox können folgende virtuelle Maschinen mit der Hilfe des Assistenten 
angelegt werden:
\begin{itemize}
        \item Windows 3.1 - 10
        \item Linux Distributionen ab dem Kernel 2.4
        \item MAC OS X
\end{itemize} 


\textbf{VMware Workstation Player:} 
 
