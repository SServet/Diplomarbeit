\chapter{Aufbau der Testumgebung}
\label{chap: Testumgebung}

\section{Virtuell oder Pysisch?}
\label{sec: VirtuellReal}

Am Anfang wurde diskutiert ob die Testumgebung (in diesem Fall ist die Testumgebung das Netzwerk wo wir das Monitoring durchführen werden) 
virtuell oder pysisch sein soll. Nach einem kurzen Meinungsaustausch aller Beteiligten wurde es entschieden dass beide Varianten verwendet werden.
 Der Grund dafür war dass wir nicht zu jedem Zeitpunkt Zugang zu der pysischenn Testumgebung hätten, aber die virtuelle Umgebung alleine nicht ausreichen 
 würde da man zum Beispiel keinen Netzwerkdrucker direkt in ein virtuelles Netzwerk integrieren kann.
 Es wurde mit dem Auftraggeber ausgemacht dass wir uns selber um das virtuelle Netzwerk kümmern und dass wir die pysische Umgebung von 
 ihm zur Verfügung gestellt bekommen.


 \section{Möglichkeiten zur Virtualisierung}
\label{sec: Verglichene Virtualiesierungsplattformen}

 Gleich am Anfang der Arbeit wurden die vier bekanntesten Virtualiesierungsplattformen vergliechen. 
 Die Plattformen die vergliechen wurden sind:

\begin{itemize}
         \item Virtualbox
         \item VMware Workstation Player
         \item VMware Workstation Pro
         \item Hyper-V
\end{itemize}

 \section{Gegenüberstellung der Virtualisierungsmöglichkeiten}
\label{sec: Gegenüberstellung der Virtualisierungsmöglichkeiten}

\textbf{Virtualbox:} Die Open-Source-Software ist durch ihre umfangreiche Ausstattung sowie ihrer 
guten Bedienerführung eines der meisten benutzten Tools zur Virtualisierung. Virtualbox unterstützt 32- 
und 64-Bit-Maschinen, die zwei gängingsten USB Standards 2.0 und 3.0, sowie USB-Laufwerke die man als 
virtuellen Datenspeicher einbinden kann, sowie das Erstellen von Snapshots (dienen zur Zustandsspeicherung eines virtuellen PCs).
Mit dem letzten Update kam auch ein bilddirektionales Drag\&Drop für die Gastsysteme. Unter Virtualbox können folgende virtuelle Maschinen 
mit der Hilfe des Assistenten angelegt werden:
\begin{itemize}
        \item Windows 3.1 - 10
        \item Linux Distributionen ab dem Kernel 2.4
        \item MAC OS X
\end{itemize} 

\noindent
\newline
\textbf{VMware Workstation Player:} Der Workstation Player ist eine kostenlose Version der Virtualisierungssoftware
von VMware. Vom Funktionumfang her liegt der Player hinter Virtualbox und kann fertige Maschinen öffnen sowie neue installieren.
Es werden fast alle Windows-Versionen sowie Linux-Variante untersützt. Der größte Nachteil vom Player ist dass hier auf 
Snapshots, sowie auf Verwaltungs- und Fernsteuerungsfunktionen verzichtet wurden. Die Schnellinstallationsmethode zum Anlegen 
virtueller Maschinen erweist sich als praktisch, da der Player die ausgewählte Installationsquelle analysiert, erkennt meist das 
verwendete Betriebssystem und übernimmt die Eingabe der Voreinstellungen wie dem Benutzernamen.

\noindent
\newline
\textbf{VMware Workstation Pro: } Die Workstation Pro ist die einzige kostenpflichtige Virtualiesierungsplattform übertrifft aber 
alle anderen Plattformen in Ausstattung, Hardware-Unterstützung, Zwischenspeicher von Snapshots, dem Kopieren und Klonen von 
virtuellen Maschinen sowie der Netzwerkkonfiguration. Ein besonderes Feature von der Pro Version ist der Snapshot-Manager, dieser
kann die Zwischenstände einfrieren, verschachteln und später zu einem gewünschten Zustand zurückkehren. Die Workstation Pro lässt 
die virtuellen Maschinen mit mehreren Nutzern teilen und gemeinsam nutzen. 

\noindent
\newline
\textbf{Hyper-V:} Hyper-V ist eine Virtualiesierungsplattform von Microsoft die seit Windows 8.1 ein Bestandteil von Windows ist 
aber nur in der 64-Bit-Version. Dieses System muss nicht wie die anderen Softwares explizit heruntergeladen werden, sondern direkt
über die Systemsteuerung von Windows aktiviert und installiert werden. Mittels Hyper-V kann man nur die Windows Versionen ab XP, die
Linux Distributionen Suse Linux Enterprise Server, Red Hat Enterprise Linux und CentOS virtualisieren, welches eine große Einschränkung 
der Auswahl ist im Vergleich zu den anderen Virtualiesierungsplattformen. Ein Vorteil vom Hyper-V ist die dynamische 
Arbeitsspeicherverwaltung dieso sorgt dafür dass beim Start einer virtuellen Maschine nur so viel vom echten RAM beansprucht wird wie 
die Maschine es braucht somit können mehrere Maschinen parallel laufen ohne dass es zu spürbaren Leistungseinbrüchen kommt. 

\section{Entscheidung}
\label{Entscheidung}

\noindent
\newline
Damit wir die für uns am besten geeignete Virtualiesierungsplattform auswählen haben 
wir vier Kriterien festegeleft die erfüllt werden müssen. Diese vier Kriterien lauten:
\begin{itemize}
        \item Kostenfrei
        \item Snapshots \newline
                Aufgrund der Gefahr dass beim Konfigurieren der Server nicht behebbare Fehler auftauchen
                können muss es mittels Snapshots eine stablile Version der Server gesichert werden können. 
        \item Ubuntu
        \item Windows 10
\end{itemize}

\begin{table}[h]
\centering
\caption{}
\begin{tabular}{ccccc}
\hline
&  $Kostenfrei$ & $Snapshots$ & $Ubuntu$ & $Windows 10$ \\ \hline
Virtualbox& \checkmark & \checkmark & \checkmark &  \checkmark \\
VMware Workstation Player&  \checkmark & X & \checkmark & \checkmark  \\
VMware Workstation Pro& X  & \checkmark & \checkmark & \checkmark \\ 
Hyper-V&  \checkmark & \checkmark & X & \checkmark \\ \hline
\end{tabular}
\end{table}

\noindent
\newline
